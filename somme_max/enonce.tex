\subsection{Sous tableau de somme maximale}

On considère un tableau d'entiers relatifs: $u=[u_0,\, \dots,\, u_{n-1}]$. \\
On note $s(i,j) = \sum_{k=i}^j u_k$. \\
Le but de cet exercice est de calculer de manière efficace: 
$$ S = \text{max}_{i,j} s(i,j)$$

\begin{enumerate}
\item Pour pouvoir tester vos fonctions des questions d'après, écrire \texttt{suite\_random(n)} qui renvoie une suite de $n$ entiers 
relatifs aléatoires.
\item On considère ici la résolution "bruteforce" du problème:
\begin{enumerate}
\item Écrire \texttt{somme\_max\_bruteforce} qui prend en entrée le tableau $u$, calcule pour tous $i,j$ la quantité $s(i,j)$ et en renvoie le max.
\item Quelle est la complexité de \texttt{somme\_max\_bruteforce} ?
\end{enumerate}
\item Implémenter \texttt{somme\_max\_bruteforce\_bis} qui calcule de manière moins coûteuse tous les $s(i,j)$. Quelle est sa complexité ?

\item On note $\sigma_j = max_{i \leq j} s(i,j)$. Comment calculer $\sigma_{j+1}$ quand on connait $\sigma_j$ ?
\item En déduire un algorithme de calcul d'un sous tableau de somme maximale en $O(n)$.
\item Implémenter cet algorithme avec une fonction \texttt{somme\_max} qui prend en entrée le tableau $u$ et qui renvoie un triplet $(S,i,j)$ avec $S$ le maximum des $s(i,j)$ et $(i,j)$ délimitant un sous-tableau de somme $S$.

\end{enumerate}
