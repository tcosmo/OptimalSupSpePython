\subsection{Le jeu de la vie}

On considère une grille 2D à $n$ lignes et $m$ colonnes. Chaque case de la grille a un état binaire: elle est soit \textit{vivante} représenté par le symbole \texttt{'\#'} ou \textit{morte} représenté par le symbole \texttt{'.'}. On appelle voisins d'une case l'une de ses huit cases adjacentes. La grille évolue selon les règles suivantes:

\begin{enumerate}
\item Si une cellule est \textit{morte} au temps $t$ et est entourée d'exactement 3 voisins \textit{vivants} elle devient \textit{vivante} au temps $t+1$.
\item Si une cellule est \textit{vivante} au temps $t$ et n'est pas entourée de deux ou trois voisins \textit{vivants} elle devient \textit{morte} au temps $t+1$.
\end{enumerate}

Le but de cet exercice est d'implémenter le jeu de la vie.

\begin{enumerate}
\item Donner l'évolution à $t+1$ des grilles:
\begin{enumerate}
\item
\begin{verbatim}
.#.
.#.
.#.
\end{verbatim}
\item
\begin{verbatim}
...
###
...
\end{verbatim}
\end{enumerate}

\item On représente les grilles en python comme des listes 2D contenant \texttt{'\#'} et \texttt{'.'}. Coder la fonction \texttt{random\_grille(n,m)} qui génère une grille aléatoire de taille $n\times m$.

\item Coder la fonction \texttt{print\_grille(grille)} qui affiche proprement la grille \texttt{grille}. \textit{Hint: Se renseigner sur "".join(list).}

\item Coder la fonction \texttt{nb\_voisins\_vivants(grille,i,j)} qui renvoie le nombre de voisins \textit{vivants} de \texttt{grille[i][j]}.

\item Coder la fonction \texttt{one\_step(grille)} qui calcule le devenir de \texttt{grille} à $t+1$.

\item Itérer \texttt{one\_step} depuis des grilles aléatoires ($10\times 50$ par exemple). Quels types de comportements observez vous asymptotiquement ?

\textit{Le jeu de la vie est un modèle dit «d'automate cellulaire» proposé par John Horton Conway en 1970. Ce modèle a été très étudié, il possède des propriétés étonnantes.}

\end{enumerate}
