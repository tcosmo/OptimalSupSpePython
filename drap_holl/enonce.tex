\subsection{Drapeau hollandais}

On se donne un tableau de réels $u$. \\
On appelle \textit{disposition hollandaise de $u$} tout tableau $v$ contenant les mêmes éléments que $u$ et qui vérifie:

\begin{align*}
\exists k_{1} \geq 0 \enspace  i < k_{1} \Rightarrow v_i < 0 \\
\exists k_{2} \geq 0 \enspace  k_{1} \leq i < k_2 \Rightarrow v_i = 0 \\
\exists k_{3} \geq 0 \enspace  k_{2} \leq i < k_{3} \Rightarrow v_i > 0 \\
\end{align*}

Exemple: pour $u=[20,-34,0,1,2,-3,0]$, $v=[-34,-3,0,0,20,1,2]$ est une disposition hollandaise de $u$.

\begin{enumerate}
\item Implémenter une fonction \texttt{drap\_holl} qui prend en entrée un tableau $u$ et qui renvoie une \textit{disposition hollandaise} de $u$.
\item Quelle est la complexité de votre implémentation ?
\item (\textbf{bonus}) Pouvez-vous faire en sorte que \texttt{drap\_holl} n'effectue qu'un seul passage dans $u$ et n'utilise pas de tableau annexe ?
\end{enumerate}

\textit{Historiquement cet exercice fait référence aux 3 couleurs du drapeau hollandais.}